\begin{table}[htbp]
\captionsetup{font=footnotesize, justification=raggedright, singlelinecheck=false}
\caption[ZT vs TPE with increasing HPO iterations]%
{Comparison of ZT and TPE on real-world datasets when TPE is allowed 
additional HPO iterations: 0, 1, 5, 10, and 20. 
At iteration 0, TPE is initialised with a random hyperparameter configuration. 
ZT represents zero-shot predictions from ZeroTune. 
"Best (\%)" indicates the percentage of datasets on which each method achieves the highest AUC, and 
"Significant (\%)" shows the fraction of datasets for which that advantage 
is statistically significant (paired t-test, \(p < 0.05\)).}
\label{table:decisiontree-zt-vs-tpe-convergence-summary}
\vskip 0.1in
\begin{center}
\begin{small}
\begin{sc}
\begin{tabular}{c c c c c}
\toprule
\multirow{2}{*}{\makecell{\textbf{HPO}\\\textbf{Iterations}}} 
& \multicolumn{2}{c}{\textbf{ZT}} 
& \multicolumn{2}{c}{\textbf{TPE}} \\
\cmidrule(lr){2-3}\cmidrule(lr){4-5}
& \textbf{Best (\%)} & \textbf{Sig (\%)} 
& \textbf{Best (\%)} & \textbf{Sig (\%)} \\
\midrule
0 & 100.0 & 100.0 & 0.0 & 0.0 \\
1 & 100.0 & 100.0 & 0.0 & 0.0 \\
5 & 70.0 & 40.0 & 30.0 & 20.0 \\
10 & 40.0 & 10.0 & 60.0 & 20.0 \\
20 & 0.0 & 0.0 & 100.0 & 60.0 \\
\bottomrule
\end{tabular}
\end{sc}
\end{small}
\end{center}
\vskip -0.1in
\end{table}